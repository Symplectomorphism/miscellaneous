\section{Solution of the Problem}
\label{sec:solution}

We have the following cases:
%
\begin{enumerate}
    \setlength\itemsep{0em}
    \item[C1.] With $p_a = \nicefrac{1}{2}$ probability, $P$ and $Q$ are sampled
        on adjacent sides.
    \item[C2.] With $p_o = \nicefrac{1}{4}$ probability, $P$ and $Q$ are sampled
        on opposite sides.
    \item[C3.] With $p_s = \nicefrac{1}{4}$ probability, $P$ and $Q$ are sampled
        on the same side.
\end{enumerate}
%
By the law of total probability, we have 
\begin{equation}
    \mathbb{P}(d \geq 1) = \mathbb{P}(d \geq 1 | \text{adjacent sides})p_a +
    \mathbb{P}(d \geq 1 | \text{opposite sides})p_o + \mathbb{P}(d \geq 1 |
    \text{same sides})p_s
    \label{eq:total}
\end{equation}
%
Let us analyze the case in which the points $P$ and $Q$ are sampled on adjacent
sides of the unit square. Then $d \geq 1$ if and only if $d^2 = x^2 + y^2 \geq
1$ if and only if $y^2 \geq 1 - x^2$. The random variable $x$ is uniformly
distributed. Thus, for a given $0 \leq x \leq 1$, we have $p(y \geq
\sqrt{1-x^2}) = 1- \sqrt{1 - x^2}$. As a result,
\begin{equation}
    \mathbb{P}(d \geq 1 | \text{adjacent sides}) = \mathbb{P}(y \geq
    \sqrt{1-x^2}) = \int_0^1 \left(1 - \sqrt{1-x^2} \right) \dd x = 1 - \int_0^1
    \sqrt{1 - x^2}\dd x = 1 - \frac{\pi}{4}.
    \label{eq:integral}
\end{equation}
%
Indeed, the integral in equation~\eqref{eq:integral} can be evaluated by
substituting $x = \sin{(u)}$ and $\dd x = \cos{(u)} \dd u$.
\begin{align*}
    \int_0^1 \sqrt{1-x^2} \dd x &= \int_0^{\nicefrac{\pi}{2}} \cos{(u)}^2 \dd u
    = \int_0^{\nicefrac{\pi}{2}} \left( \frac{1}{2}\cos{(2u)} + \frac{1}{2}
    \right) \dd u = \left[\frac{1}{4}\sin{(2u)} +
    \frac{u}{2}\right]_{u=0}^{u=\nicefrac{\pi}{2}} = \frac{\pi}{4}.
\end{align*}
%
% Using the above computation to evaluate the definite integral in
% equation~\eqref{eq:integral}, we obtain
% \[ \mathbb{P}(y \geq \sqrt{1-x^2}) = 1 - \frac{\pi}{4}. \]
%
The remaining terms in equation~\eqref{eq:total}, namely, $\mathbb{P}(d \geq 1
|\text{opposite sides})$ and $\mathbb{P}(d \geq 1 | \text{same sides})$ are
readily seen to equal $1$ and $0$, respectively. Plugging in the appropriate
values in equation~\eqref{eq:total} thus yields
%
\begin{align*}
    \mathbb{P}(d \geq 1) &= \mathbb{P}(d \geq 1 | \text{adjacent sides})p_a +
    \mathbb{P}(d \geq 1 | \text{opposite sides})p_o + \mathbb{P}(d \geq 1 |
    \text{same sides})p_s \\ &= \nicefrac{1}{2}\left(1 -
    \nicefrac{\pi}{4}\right) + \nicefrac{1}{4} \times 1 + \nicefrac{1}{4} \times
    0 = \boxed{\nicefrac{3}{4} - \nicefrac{\pi}{8}}\approx 35.73\%.
\end{align*}
