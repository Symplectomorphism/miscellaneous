\section{Sorunun \c{C}\"{o}z\"{u}m\"{u}}
\label{sec:solution}

Elbette bu sorunun cozumunu yapmak icin deneme yanilma ile ilerleyebiliriz. Bu
dokumani yazmamdaki amacimiz, bu soruyu nasil bir karma tamsayi eniyileme
sorusuna (mixed-integer optimization problem) donusturebilecegimizi ve bu tip
eniyileme sorularini cozen programlari kullanarak nasil cevabi bulabilecegimizi
aciklamaktir.

Sorunun cozumune baslamak icin once uc tane degisken tanimlayalim: 
\begin{enumerate}
    \item $\mathbb{N} \ni t_k $: Sonaj'in $k^{\text{inci}}$ adimda icinde bulundugu zaman (yil ve tamsayi),
    \item $\mathbb{N} \ni a_k$: Sonaj'in $k^{\text{inci}}$ adimdaki yas (tamsayi),
    \item $\mathbb{R} \ni s_k$: Sonaj'in $k^{\text{inci}}$ adimda elinde bulunan kaynak (mg ve gercel sayi).
\end{enumerate}
%
Bunlara ek olarak Sonaj $k^{\text{inci}}$ adimda $3$ secimle karsi karsiya
kaliyor: (i) zamanda 33 yil ileri yolculuk etmek, (ii) 11 yil beklemek, (iii)
zamanda 33 yil geri yolculuk etmek. Bu secimi gostermek icin de
$\{u_{kj}\}_{j=1}^3 \in \{0, 1\}$ simgesini kullanalim, oyle ki
%
\begin{equation}
    \sum_{j=1}^3 u_{kj} = 1,
    \label{eq:constraints_u}
\end{equation}
%
olsun. Demek ki $u_k$ Sonaj'in $k^{\text{inci}}$ adimdaki kararini temsil eden
bir $3$-vektor. Her elemani ya sifir ya da bir olacak, ve yukaridaki toplami
saglamak icinse tam olarak bir elemani bir olmak zorunda. Bu vektoru de soyle
kodlayalim: eger $u_{k1} = 1$ ise Sonaj $33$ yil ileri, $u_{k3} = 1$ ise $33$
yil geri yolculuk ediyor olsun. $u_{k2} = 1$ olursa da Sonaj $11$ yil bekliyor
olsun.

Diyelim ki birinci adimdan basliyoruz, yani $k=1$'den basliyor. Bu noktada $k$
ile dizinledigimiz adimlar sayisinin kaca kadar gideceginizi daha bilmiyoruz. Bu
sayiya simdilik $N$ diyelim. Fakat $N$ hakkinda soyle bir bilgimiz var: Sonaj
her zaman yolculugundan sonra 11 yil beklemek zorunda ve kaynak (Secium 731) her
11 yilda bir yarilaniyor. Zaman yolculugu yapmak icin de kaynaktan en az 1 birim
bulunmasi gerekiyor. Hal boyleyken kac yarilanmadan sonra zaman yolculugu icin
Sonaj'in elinde yeterince kaynak kalmaz? Hemen hesaplayalim:
%
\begin{equation*}
    \frac{s_1}{2^m} \leq 1 \Rightarrow m \leq \log_2{(s_1)}.
\end{equation*}
%
Yani, Sonaj $t=0$'a geri donmek istiyorsa en fazla $\log_2{(s_1)}$ yarilanmaya
tahammul edebilir. Soruda bize, kaynagin baslama degeri $s_1 = 21$ olarak
verilmis. Her yarilanma arasinda da bir yolculuk yapabilecegine gore, toplam
adim sayisina su ust siniri dayatabiliriz:
\[ N \leq 2 \left(\lceil \log_2{(s_1)} \rceil\right) \approx 2\left(\lceil 4.39
\rceil\right) = 10. \]
%
Bundan sonra stratejimiz soyle olacak: $N$'i sifirdan baslatacagiz (hic adim
atmama durumu) ve $N=10$'a kadar ayni eniyileme sorusunu cozup, buldugumuz en
iyi degeri dogru cevap olarak ilan edecegiz.

Son olarak iki tane daha eniyileme degiskenine ihtiyacimiz var. Bunlari
$\mathbb{N} \ni T$ (tamsayi) ve $\left\{y_k \in \{0, 1\}:k = \{1, \ldots, N\}
\right\}$ olarak gosterecegiz. Bu degiskenleri tanitmamizin nedeni de ``Sonaj'in
zamanda gorebilecegi en ileri tarihi gormus'' olmasini matematiksel olarak ifade
etme istegimizdir. Simdi, bizim $t$ vektorumuz her $k \in \underline{N} := \{1,
\ldots, N\}$ icin Sonaj'in $k^{\text{inci}}$ adimda icinde bulundugu yili temsil
etmekte. Amacimiz bu vektorun en buyuk elemanini maksimize etmek. Bunu
matematiksel eniyileme teorisini uygulamaya en elverisli sekilde ifade etmemiz
lazim ki eniyileme programimiz soruyu cozebilsin. Bunu soyle elde ediyoruz.
Oncelikle $y$ vektorunun elemanlarinin toplamini $1$ yapiyoruz ki sadece ve
sadece bir elemani $1$ degerini alsin, gerisi $0$ olsun: 
\begin{equation}
    \sum_{k=1}^N y_k = 1. 
    \label{eq:constraints_y}
\end{equation}
%
Sonra, $T$ degiskenini oyle sececegiz ki, eniyileme programi bu degeri
$t$-vektorunun en buyuk elemanindan ufak yapmak isteyecek. Bunu da eniyileme
literaturunde big-$M$ yontemini~\citep{griva2009linear,wiki:Big_M_method}
kullanarak yapacagiz. Yontem buyuk ve sabit bir $M$ degeri secerek basliyor. Bu
deger her soruya gore degisik olacaktir. Biz bu problem icin $M=100$ degerinin
calistigini gozlemledik. Once bu yontemin bize verdigi kisita goz atalim, sonra
neden bu kisitin calistigini anlatalim.
%
\begin{equation}
    T - t_k \leq M\left( 1 - y_k \right), \qquad k = 1, \ldots, N.
    \label{eq:T_kisiti}
\end{equation}
%
Elbette eniyileme amacimiz $T$'yi maksimize etmek olacak.
Denklem~\eqref{eq:T_kisiti}'deki kisit neden calisiyor? Hemen anlamaya
calisalim. Oncelikle hatirlayalim ki $y_j$'lerden sadece biri $1$ degerini,
gerisi $0$ degerini alacak. Diyelim ki $y_m = 1$ ve geri kalan $y_j = 0$ ve $t$
vektorumuz oyle ki gercekten de $t_m \geq t_j, \; \forall j \neq m$. Tabi ki $1
\leq j, m \leq N$. Bu durumda yukaridaki kisitimiz soyle cozunuyor:
%
\begin{align*}
    T &\leq t_m, \\
    T &\leq t_j + M, \quad j \neq m.
\end{align*}
%
Gordugumuz gibi bu gercekten de istedigimiz davranis sekli. $M$ buyuk bir sabit
oldugu icin ikinci satirdaki esitsizlikler eniyileme sorumuzu kotu-tanimli
yapmazken, birinci satirdaki esitsizlik eniyileme programinin $T$'nin seciminde
$T$'yi $t_m$'ye kadar yukseltme olanagi sunuyor. 

Peki ya $y_m = 1$ degil de $y_n = 1$ ve $n \neq m$, yani $t_n \leq t_m$ olsaydi
ne olacakti? O zaman hala bu insa bize istedegimizi verecek miydi? Hemen analiz
edelim. Bu durumda~\eqref{eq:T_kisiti}'deki kisitlar su sekilde cozunuyor:
%
\begin{align*}
    T &\leq t_n, \\
    T &\leq t_j + M, \quad j \neq n.
\end{align*}
%
Hal boyle olunca, eniyileme programi $T$ degerini maksimize ederken en fazla
$t_n$'nin degerine kadar cikarabilir. Oysa $t_m$'nin degeri $t_n$'den daha
buyuk! Korkmamiza gerek yok, cunku $y$ vektoru eniyileme programinin kontrolu
altinda. Bu durumda $y_n = 1$ degil de $y_m = 1$ secimini yaptiginda daha iyi
sonuc elde edecegini eniyileme programinin bulgulayicisi (heuristic)
hesaplayacaktir. Bunun icin elimizde iyi teorik yontemler
var~\cite{griva2009linear}.

Soruyu eniyileme programina devretmeden once birkac kisit daha koymamiz
gerekiyor cunku $u_{kj}$'in degerine gore $a_{k+1}$, $s_{k+1}$, ve $t_{k+1}$'in
alacagi degerler uzerinde bazi kisitlar var ve onlari daha modellemedik.
Sonaj'in yasini simgeleyen $a$ ile baslayalim. O zaman,
%
\begin{equation*}
    a_{k+1} = 
    \begin{cases}
        a_k + 11 & \mbox{ eger } u_{k2} = 1, \\
        a_k & \mbox{ eger } u_{k1} + u_{k3} = 1.
    \end{cases}
    % \label{eq:a_constraints_if_else}
\end{equation*}
%
Tabi bu denklemi bu sekilde eniyileme programina sunamayiz. Kosullu aciklamalari
matematiksel olarak programa anlatmamiz gerekmekte. Bunu da yine ``big-$M$''
yontemini kullanarak elde edebiliriz. Her $k \in \underline{N}$ icin su
kisitlari dayatalim:
%
\begin{align}
    \begin{split}
    a_{k+1} &\leq a_k + 11 + M\left( 1 - u_{k2} \right), \\
    a_{k+1} &\geq a_k + 11 - M\left( 1 - u_{k2} \right), \\
    a_{k+1} &\leq a_k + M\left( 1 - u_{k1} - u_{k3} \right), \\
    a_{k+1} &\geq a_k - M\left( 1 - u_{k1} - u_{k3} \right).
    \end{split}
    \label{eq:a_constraints_big_M}
\end{align}
%
Bu kisitlarin ustunde biraz dusunursek neden calistiklarini anlayacagiz.
Oncelikle $k^{\text{inci}}$ adimda Sonaj'in $11$ yil bekleme karari aldigini
dusunelim. Yani $u_{k2} = 1$ olsun. Bu durumda ilk iki esitsizlik bir araya
gelip bize istedigimiz $a_{k+1} = a_k + 11$ denklemini veriyor. Ucuncu ve
dorduncu esitsizlikler ise asagidakini donusuyor: 
\[
    a_k - M \leq a_{k+1} \leq a_k + M, 
\]
%
Gordugumuz gibi $M$ buyuk bir sayi oldugu icin bu esitsizlikler herhangi bir
tutarsizliga neden olmuyor. Simdi diger durumu dusunelim, yani $u_{k1} = 1$
olsun. Bu durumda ucuncu ve dorduncu denklem bize $a_{k+1} = a_k$ verirken, ilk
iki denklem de yine $M$'nin buyuk degerinden dolayi tutarsizlik yaratmiyor: 
\[ a_k + 11 - M \leq a_{k+1} \geq a_k + 11 + M. \]
%
\begin{rem}
    Dikkatli olalim, eger $M$'yi $11$ degerinden kucuk secseydik, eniyileme
    programi $a_{k+1} = a_k$'yi gecerli sayamazdi ve dolayisiyla elimizdeki
    soruyu duzgun ifade etmemis olurduk.
\end{rem}

Artik ``big-$M$'' yontemini kullanmayi ogrendik. Her $k \in \underline{N}$ icin
$s_{k+1}$ uzerindeki kisitlari da bu yontemi kullanarak yazalim.
%
\begin{align}
    \begin{split}
        s_{k+1} &\leq \nicefrac{s_k}{2} + M\left( 1 - u_{k2} \right), \\
        s_{k+1} &\geq \nicefrac{s_k}{2} - M\left( 1 - u_{k2} \right), \\
        s_{k+1} &\leq s_k + 1 + M\left( 1 - u_{k1} - u_{k3} \right), \\
        s_{k+1} &\geq s_k + 1 - M\left( 1 - u_{k1} - u_{k3} \right).
    \end{split}
    \label{eq:s_constraints_big_M}
\end{align}
%
Ayni sekilde, $t_{k+1}$ uzerindeki kisitlari da su sekilde ifade edebiliriz.
%
\begin{align}
    \begin{split}
        t_{k+1} &\leq t_k + 33 + M\left( 1 - u_{k1} \right), \\
        t_{k+1} &\geq t_k + 33 - M\left( 1 - u_{k1} \right), \\
        t_{k+1} &\leq t_k + 11 + M\left( 1 - u_{k2} \right), \\
        t_{k+1} &\geq t_k + 11 - M\left( 1 - u_{k2} \right), \\
        t_{k+1} &\leq t_k - 33 + M\left( 1 - u_{k3} \right), \\
        t_{k+1} &\geq t_k - 33 - M\left( 1 - u_{k3} \right).
    \end{split}
    \label{eq:t_constraints_big_M}
\end{align}
%
Simdiye kadarki butun argumanlarimizi bir araya getirince ve soruda verilen
sinir sartlarini da sorumuza ekleyince, eniyileme programimizi sonunda tam
olarak yazabiliriz. Hatirlayalim ki, $N$'yi $1$'den baslatip $10$ degerine kadar
bu eniyileme programimizi tekrar cozecegiz.
%
\begin{equation}
    \begin{aligned}
        \underset{a,\, s, \, t, \, u, \, y, \, T }{\textrm{maximize}} && T &, \\
        \textrm{subject to} 
        &&\quad (1) &-(6), \\
        &&\quad t_1 &= t_{N+1} = 0, \\
        &&\quad a_1 &= 15, \\
        &&\quad s_1 &= 21.  \\
    \end{aligned}    
    \label{eq:optimization}
\end{equation}