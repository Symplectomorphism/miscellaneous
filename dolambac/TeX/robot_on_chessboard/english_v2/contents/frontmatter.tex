\begin{frontmatter}

%% Title, authors and addresses

%% use the tnoteref command within \title for footnotes;
%% use the tnotetext command for theassociated footnote;
%% use the fnref command within \author or \affiliation for footnotes;
%% use the fntext command for theassociated footnote;
%% use the corref command within \author for corresponding author footnotes;
%% use the cortext command for theassociated footnote;
%% use the ead command for the email address,
%% and the form \ead[url] for the home page:
%% \title{Title\tnoteref{label1}}
%% \tnotetext[label1]{}
%% \author{Name\corref{cor1}\fnref{label2}}
%% \ead{email address}
%% \ead[url]{home page}
%% \fntext[label2]{}
%% \cortext[cor1]{}
%% \affiliation{organization={},
%%            addressline={}, 
%%            city={},
%%            postcode={}, 
%%            state={},
%%            country={}}
%% \fntext[label3]{}

\title{Mischievous Sibling's Grid World}

%% use optional labels to link authors explicitly to addresses:
%% \author[label1,label2]{}
%% \affiliation[label1]{organization={},
%%             addressline={},
%%             city={},
%%             postcode={},
%%             state={},
%%             country={}}
%%
%% \affiliation[label2]{organization={},
%%             addressline={},
%%             city={},
%%             postcode={},
%%             state={},
%%             country={}}

\author[label1]{Aykut C. Satici}
\ead{aykutsatici@boisestate.edu}

% \author[label2]{Gokhan Atinc}
% \ead{gokhan.atinc@mathworks.com}

%Department and Organization
\affiliation[label1]{organization={Boise State University},
            addressline={Mechanical and Biomedical Engineering}, 
            city={Boise},
            postcode={83725}, 
            state={Idaho},
            country={USA}}

% \affiliation[label2]{organization={Mathworks Inc.},
%             addressline={1 Lakeside Campus Drive}, 
%             city={Natick},
%             postcode={01760}, 
%             state={Massachusetts},
%             country={USA}}

\begin{abstract}
    This is a technical note solving an interesting question posed by a friend 
    and colleague, G\"{o}khan At{\i}n\c{c}. The question starts out
    exactly like the well-known Grid World problem, but it has a twist. We solve
    the problem in two different ways: first using a direct probability
    calculation and then using reinforcement learning on a judiciously designed
    Markov Decision Process. The results are compared and discussed.
\end{abstract}

% %%Graphical abstract
% \begin{graphicalabstract}
% %\includegraphics{grabs}
% \end{graphicalabstract}
% 
% 
% %%Research highlights
% \begin{highlights}
% \item Research highlight 1
% \item Research highlight 2
% \end{highlights}


\begin{keyword}
Probability \sep Expectations \sep Reinforcement learning \sep Gymnasium \sep
Policy iteration 

% \JEL C45 \sep C53 \sep C55 \sep C82 \sep F47 \sep O50
%% keywords here, in the form: keyword \sep keyword
%% PACS codes here, in the form: \PACS code \sep code
%% MSC codes here, in the form: \MSC code \sep code
%% or \MSC[2008] code \sep code (2000 is the default)
\end{keyword}


\end{frontmatter}