\section{Introduction}
\label{sec:intro}
%
Grid World environments are frequently utilized to establish baselines for many
novel reinforcement learning algorithms. These environments are simpler compared
to other discrete environments such as chess and go; they have much smaller
state and action spaces. Yet, they still require reasoning for a sequential,
evaluative and sampled feedback problem. In this sense, these set of problem
provide a good entry point to either test new algorithms or start learning basic
concepts of reinforcement learning.

In this note, we consider the ordinary instance of a stochastic grid world
environment, in which the problem is to find the shortest path from a start
state to a goal state. However, in this case, the task is to do this in a
one-shot manner. That is, the agent must learn to find the optimal policy in a
single episode. This is a challenging problem, as the agent must learn to
explore the environment, and exploit the information it has gathered to reach
the goal state in the fewest number of steps possible. Furthermore, the
reinforcement learning problem we pose to solve this instance of the grid world
problem requires learning two $Q$-functions as opposed to one; one corresponding
to a familiar $Q$-function for a classical grid world, and the other a
$Q$-function for a related multi-armed bandit problem. The agent must learn to
balance the exploration of the grid world environment with the exploitation of
the multi-armed bandit problem to solve the task optimally. We coin the term 
``sibling's grid world'' to describe this problem.

The contribution of this paper is in introducing a baseline grid world
environment in which one-shot RL algorithms may be evaluated, providing a 
theoretical solution to this problem, and comparing the performance of a 
baseline $Q$-learning RL solution to the theoretical solution.

We provide a thorough statement of this problem in Section~\ref{sec:problem},
before providing a theoretical solution in the first subsection of
Section~\ref{sec:solutions}. The subsequent subsection of the same section then
provides our formulation of the reinforcement learning problem to solve the
sibling's grid world in a one-shot manner. We then present the results of our RL
formulation in Section~\ref{sec:experiment}, comparing and contrasting against 
the theoretical solution in Section~\ref{ssec:prob_sol}. We conclude with a
discussion of the results and potential future work in Section~\ref{sec:conclusion}.