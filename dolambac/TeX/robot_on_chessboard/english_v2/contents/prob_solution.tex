\subsection{Probability Solution}
\label{ssec:prob_sol}
%
We invoke a celebrated theorem from probability theory that will help us solve
the problem. This theorem is called the \textit{law of total expectation} or
\textit{tower rule of probability theory} in the
literature~\cite{bertsekas2002introduction}.
%
In this section, we use the correspondence $\{1, 2, 3, 4, 5\} \mapsto \{0, 1, 2,
3, 4\}$ and $\{A, B, C, D, E\} \mapsto \{0, 1, 2, 3, 4\}$ for ease of
exposition.
%
\begin{thm}[Tower rule] \label{thm:tower} Let $X$ be a random variable and
    $\{A_i\}_i^m$ be a finite partition of the sample space. Then, 
    \[ \mathbb{E}[X] = \sum_i^m \mathbb{E}\left[ X \mid A_i \right]
    \mathbb{P}(A_i). \]
\end{thm}
%
We denote the square on which the robot resides at the $k^{\text{th}}$ step by
$s_k$ and the goal square by $g$. Consider the following family of random
variables.
%
\begin{equation}
    C_n := \sum_{k=n+1}^\infty r_k, \qquad r_k = 
\begin{cases}
    1 & \mbox{if } s_k \neq g \\
    0 & \mbox{if } s_k = g
\end{cases}.
\label{eq:RVs}
\end{equation}
%
The sum in this definition is well-defined because once the robot reaches $g$,
all the remaining $r_k$'s take the value zero. We are being asked the value of
$\mathbb{E}[C_0]$. To that end, we will use the following repercussion of
definition~\eqref{eq:RVs}.
% \vspace{-2mm}
\begin{equation*} 
    C_m = \sum_{i=m+1}^n r_i + C_n, \qquad m \leq n.
\end{equation*}
% \vspace{-3mm}
% Beklenen de\u{g}erin \"{o}zelliklerini kullanarak a\c{s}a\u{g}{\i}daki sonuca
% varabiliriz.
Using the properties of expectation, if $s_i \neq g$ for $m < i \leq n$, we can
deduce the following result.
% \vspace{-1mm}
\begin{equation*} 
    \mathbb{E}[C_m] = \sum_{i=m+1}^n \mathbb{E}[r_i] + \mathbb{E}[C_n] = n-m + \mathbb{E}[C_n].
    % \begin{cases}
    %     k + \mathbb{E}[C_n] & \mbox{e\u{g}er } s_i \neq g \\
    %     \mathbb{E}[C_n] & \mbox{e\u{g}er } s_i = g
    % \end{cases}, 
    % \qquad m < i < n, \;\; s_i \neq g.
\end{equation*}
%
Similar identities to the above expression hold for conditional expectations as
well. Let us start to compute the expectation of the ``value'' of being in state
$C3$ using the tower rule~\ref{thm:tower}. Of course, by ``value,'' we mean the
expected minimum number of times a button needs to be pressed in order to reach
the goal state.
% %
\begin{align}
   \begin{split}
    \mathbb{E}[C_0] &= \mathbb{E}\left[ C_0 \mid s_1 \in \{C4, D3\} \right] 
    \underbrace{\mathbb{P}(s_1 \in \{C4, D3\})}_{=\frac{1}{2}} \\ 
    &+
    \mathbb{E}\left[ C_0 \mid s_1 \in \{B3, C2\} 
    \right] \underbrace{\mathbb{P}(s_1 \in \{B3, C2\})}_{=\frac{1}{2}}.
    \end{split}
    \label{eq:tower1}
\end{align}
%
From now on, we will assume $s_1 = C4$ in the first term on the right-hand side
and $s_1 = C2$ in the second. Notice that, because of the inherent symmetry of
the problem, while the state $s_1 = D3$ is equivalent to $s_1 = C4$; $s_1 =
B3$ is equivalent to $s_1 = C2$ in terms of their value.

\paragraph{Computation of the first conditional expectation
in~\eqref{eq:tower1}} 
% \"{O}nce~\eqref{eq:tower1}'inci denklemin sa\u{g} taraf{\i}ndaki ilk
% beklenen de\u{g}er terimini ele alaca\u{g}{\i}z. Kule kural{\i}n{\i} kullanarak
% hesaplamalar{\i}m{\i}za devam edelim. 
First of all, we consider the first expectation term on the right-hand side of
equation~\eqref{eq:tower1}. If at the first step, we found ourselves in square 
$C4$, that means we made progress towards the goal. We can keep this greedy 
motion for one more turn since it will get us closer to the goal. Hence, $s_2 = 
C5$, with a cost of $r_2 = 1$. Since at step $2$, we hit the south wall, we need
to change the button to press. We have $3$ options of equal likelihood. Choosing
one will take us either to $D5$, back to $C4$, or to $B5$. Hence,
%
\begin{align}
    \begin{split}
    \mathbb{E}&\left[ C_0 \mid s_1 = C4 \right] = \overbrace{r_1 + r_2}^{=2} 
    \\ + &\overbrace{\mathbb{E}[C_2 \mid s_1 = C4, s_3 = D5]}^{=2} \overbrace{\mathbb{P}(s_3 = D5 \mid s_1 = C4)}^{=\frac{1}{3}} 
    \\ + &\mathbb{E}[C_2 \mid s_1 = s_3 = C4] \underbrace{\mathbb{P}(s_3 = C4 \mid s_1 = C4)}_{=\frac{1}{3}} \\
    + &\mathbb{E}[C_2 \mid s_1 = C4, s_3 = B5] \underbrace{\mathbb{P}(s_3 = B5 \mid s_1 = C4)}_{=\frac{1}{3}}.
    \end{split}
    \label{eq:tower2}
\end{align}
%
We delve further into the computation of the two expected values in
equation~\eqref{eq:tower2}, whose values are not immediately apparent. If at 
step $3$ we found ourselves back on $C4$, we got farther from the goal, but 
we can repeat our previous move to get back to $C5$ on step $4$. This maneuver 
costs us $r_3 + r_4 = 2$ units. Now, we know what directions two of the keys map to. The remaining expectations are computed below.
%
\begin{align*}
    \begin{split}
    \mathbb{E}&[C_2 \mid s_1=s_3=C4] = \underbrace{r_3+r_4}_{=2}
    \\ + &\underbrace{\mathbb{E}[C_4 \mid s_1=s_3=C4, s_5=D5]}_{=2} \underbrace{\mathbb{P}(s_5=D5 \mid s_1=s_3=C4)}_{=\frac{1}{2}} \\
    + &\underbrace{\mathbb{E}[C_4 \mid s_1=s_3=C4, s_5=B5]}_{=4} \underbrace{\mathbb{P}(s_5=B5 \mid s_1=s_3=C4)}_{=\frac{1}{2}}.
    \end{split}
\end{align*}
%
There is only one expected value that we have left to compute in
equation~\eqref{eq:tower2}. Again, if we found ourselves on the $B5$ square at 
step $3$, we have moved farther away from the goal, but do now know how to get 
back to $C5$ in this case. Hence, we need to try out the final key to figure 
out the full mapping. The expectations are computed below.
%
\begin{align*}
    \begin{split}
    \mathbb{E}&[C_2 \mid \; s_1=C4, s_3=B5] = \\
    &\underbrace{\mathbb{E}[C_2 \mid s_1=C4, s_3=B5, s_4=C5]}_{=4} \\ &{\phantom{1234}} \times \underbrace{\mathbb{P}(s_4=C5 \mid s_1=C4, s_3=B5)}_{=\frac{1}{2}} \\
    &+ \underbrace{\mathbb{E}[C_2 \mid s_1=C4, s_3=B5, s_4=B4]}_{=6} \\ &\phantom{1234} \times \underbrace{\mathbb{P}(s_4=B4 \mid s_1=C4, s_3=B5)}_{=\frac{1}{2}}.
    \end{split}
\end{align*}
%
The computations above allows us to determine the first expected value in
equation~\eqref{eq:tower1} using equation~\eqref{eq:tower2}:
\fbox{$\mathbb{E}[C_0 \mid s_1=C4] = 6$}.

\paragraph{Computation of the second conditional expectation
in~\eqref{eq:tower1}} 
%
We use similar techniques to compute the second expected value on the right-hand
side of equation~\eqref{eq:tower1}. If on step $2$, we find ourselves on square 
$B2$, then, we have identified all actions that take us farther from the goal. 
The two remaining actions both get us closer to the goal, from which we are $8$
steps away.
%

\begin{align}
    \begin{split}
    \mathbb{E}&\left[ C_0 \mid s_1 = C2 \right] = \\
    &\underbrace{\mathbb{E}[C_0 \mid s_1 = C2, s_2 = B2]}_{=8} \underbrace{\mathbb{P}(s_2 = B2 \mid s_1 = C2)}_{=\frac{1}{3}} \\
    &+ \mathbb{E}[C_0 \mid s_1=C2, s_2 = C3] \underbrace{\mathbb{P}(s_2 = C3 \mid s_1 = C2)}_{=\frac{1}{3}} \\
    &+ \mathbb{E}[C_0 \mid s_1 = C2, s_2 = D2] \underbrace{\mathbb{P}(s_2 = D2 \mid s_1 = C2)}_{=\frac{1}{3}}.
    \end{split}
    \label{eq:tower3}
\end{align}
%
Once more, we utilize the tower rule to compute the expected values in
equation~\eqref{eq:tower3} whose values are not immediately apparent. If at step
$3$, we end up on square $C3$, we can keep repeating this move until we reach
$C5$ at step $4$, incurring a cost of $r_1 + r_2 + r_3 + r_4 = 4$ units. From
here, the only two possibilities is we get to square $D5$ or $B5$ on step $5$.
If we reach square $D5$, then the same motion will get us to the goal, hence the
corresponding expected value is $2$. If, on the other hand, we end up on square
$B5$ on move $5$, we have moved farther away from the goal, but now we know
which button to press to get to the goal, resulting in an expected value of $4$.
%
\begin{align*}
    \begin{split}
    \mathbb{E}[C_0 \mid \; &s_1=C2, s_2=C3] = \overbrace{r_1 + r_2 + r_3 + r_4}^{=4} \\
    &+ \underbrace{\mathbb{E}[C_4 \mid s_1=C2, s_2=C3, s_5=D5]}_{=2} \\ &{\phantom{1234}}\times \underbrace{\mathbb{P}(s_5=D5 \mid s_1=C2, s_2=C3)}_{=\frac{1}{2}} \\
    &+ \underbrace{\mathbb{E}[C_4 \mid s_1=C2, s_2=C3, s_5=B5]}_{=4} \\ &{\phantom{1234}} \times \underbrace{\mathbb{P}(s_5=B5 \mid s_1=C2, s_2=C3)}_{=\frac{1}{2}}.
    \end{split}
\end{align*}
%
There is only one expected value that we have left to compute in
equation~\eqref{eq:tower3}. If we find ourselves on square $D2$ at step $2$, we
can repeat this direction to reach $E2$ at step $3$, incurring a cost of $r_1 +
r_2 + r_3 = 3$ units. From here, we can either go to $E3$ or $D2$ with the
remaining keys. If we find ourselves on $E3$, we can repeat the same motion to
reach the goal in $3$ more steps. If we find ourselves on $D2$ at step $4$, we
need to get back to $E2$ and use the remaining key to reach the goal, incurring
a cost of $5$ units.
%
\vspace{-2mm}
\begin{align*}
    \begin{split}
    \mathbb{E}[C_0 \mid \; &s_1=C2, s_2=D2] = \overbrace{r_1 + r_2 + r_3}^{=3} \\
    &+ \underbrace{\mathbb{E}[C_3 \mid s_1=C2, s_2=D2, s_4=E3]}_{=3} \\ &{\phantom{1234}}\times \underbrace{\mathbb{P}(s_4=E3 \mid s_1=C2, s_2=D2)}_{=\frac{1}{2}} \\
    &+ \underbrace{\mathbb{E}[C_3 \mid s_1=C2, s_2=s_4=D2]}_{=5} \\ &{\phantom{1234}}\times \underbrace{\mathbb{P}(s_4=D2 \mid s_1=C2, s_2=D2)}_{=\frac{1}{2}}.
    \end{split}
\end{align*}
%
We can now use equation~\eqref{eq:tower3} in order to compute the second
expected value on the right-hand side of
equation~\eqref{eq:tower1}:\fbox{$\mathbb{E}[C_0 \mid s_1=C2] = \frac{22}{3}$}.
%
We have computed every quantity that we are interested in. Now, we go back to
equation~\eqref{eq:tower1}, and plug in the values we have computed to find the
expected value of the minimum number of times Alice needs to press the buttons. 

\begin{empheq}[box=\widefbox]{align}
    \begin{split}
    \mathbb{E}[C_0] = &\underbrace{\mathbb{E}\left[ C_0 \mid s_1 \in \{C4, D3\} \right]}_{=6}
    \underbrace{\mathbb{P}(s_1 \in \{C4, D3\})}_{=\frac{1}{2}}\\ 
    &+ \underbrace{\mathbb{E}\left[ C_0 \mid s_1 \in \{B3, C2\} \right]}_{=\frac{22}{3}} 
    \underbrace{\mathbb{P}(s_1 \in \{B3, C2\})}_{=\frac{1}{2}} \\ &= \frac{20}{3} = 6.\bar{6}. \nonumber
    \end{split}
    % \label{eq:tower}
\end{empheq}