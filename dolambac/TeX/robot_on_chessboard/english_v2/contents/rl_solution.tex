\subsection{RL solution}
\label{ssec:rl_sol}
%
We want to be able to programmatically solve this toy problem so that we can 
find the solution for any given initial state. We can use reinforcement learning
(RL) to solve this problem. 

The reasoning goes as follows: we want to apply RL techniques, such as double
$Q$-learning~\cite{morales2020grokking} to find the action-value function and
from it the state-value function. However, in order to do this, we need to
define the state, action, transition, and reward functions. The key to solving 
this problem efficiently with RL methods is to observe that the state does not 
only comprise the position of the robot on the chessboard, but also the ``world 
belief''. Here, we define world belief as our current belief of mapping of the 
arrow keys to the actual directions.

Let $\uline{N} = \{0, 1, \ldots, N-1\}$ denote the set of nonnegative integers
ranging from $0$ to $N-1$. We model this problem as consisting of two RL 
subproblems: the first one is the sequential problem of finding the shortest 
route to the goal position, given the world belief, and the second is to 
find an appropriate world belief, which can be modeled as a $24$-armed bandit 
problem.

We define the Markov decision processes (MDP) of the subproblems as follows:

\subsubsection{State Spaces}
%
The first subproblem has the state space $\mc{S}_1 = \uline{4} \times
\uline{4}$, where the first factor stands for the column in which the robot 
is located, and the second stands for the row.
%
The second subproblem is a bandit problem, with a singleton state, the world 
belief.
%
The full problem may be considered to have the state space $\mc{S} = \mc{S}_1 \times \{1\} \cong \uline{4} \times \uline{4}$.

\subsubsection{Action Spaces}
%
The problem has the action space $\mc{A} = \uline{4} \times \uline{24}$. The 
first factor corresponds to the which key to be pressed given the world belief, 
which is an element of the second factor of the action space.

\subsubsection{Transition function}
%
The transition function is a deterministic function of the true world. The true 
world has a certain mapping between the keys and the directions, which is hidden
from the agent. This key mapping is deterministic albeit hidden and the agent
will move in the corresponding directions given the correct key mapping one
hundred percent of the time.

\subsubsection{Reward function}
%
This function needs to be designed judiciously in order to give the agent the
best chance to learn the optimal strategy. What worked well in our experiments
is to assign a reward of $-1$ each time the agent moves into a nonterminal state
(a state that is not $G$) for the grid world subproblem. 

For the bandit subproblem, we define the reward as a sum of two terms. The first
term is the difference between the value functions of the gridworld problem,
given that the current world belief is correct. To explain, suppose that $w \in
\uline{24}$ is the current world belief. We perform value iteration to find the
value function of the grid world under this assumption and we assign the first
term of the reward for the bandit as the value function evaluated at the grid
world position at which we land minus the value function evaluated at the
current grid world position.

It turns out that it is important to be able to distinguish between whether the 
desired movement direction is equal to the true direction of motion. For that 
reason, the second term that we add to the bandit reward signal $-2$ whenever
the executed direction is different from the direction in which the agent
expected the robot to move.

Our implementation~\cite{saticiGitHub} of this MDP is carried out using 
Gymnasium~\cite{towers2024gymnasium}, a Python library for modeling and RL.


\subsubsection{Q-Learning}

Once the MDP is defined, we apply $Q$-learning to find the optimal policy. For 
the grid world subproblem, this is the straightforward $Q$-learning algorithm 
that can be found in any RL textbook. For the bandit subproblem, the $Q$-function is learned by the iteration
%
\[
\mathtt{
Q_b[a] = Q_b[a] + (reward - Q_b[a]) / N[a],
}
\]
%
where $\mathtt{N[a]}$ is the number of times action $a$ has been taken for the 
bandit problem.

The way we set this problem up, allows us to use RL to learn to get to the goal
position in one episode in the last number of moves. Of course, this is a
stochastic process, depending on what the initial world belief is selected and
what the true world is. We perform Monte Carlo simulations of this setup to determine the statistics, including the expectation of the minimum number of 
moves (which was solved analytically in Section~\ref{ssec:prob_sol}).