\section{Background}
\label{sec:background}

In this section, we summarize the technical background that is used in the
remainder of the paper. Additional details on the cursory exposition here, may be
found in~\citet{ortega2002stabilization,van2000l2} and references therein.

% \subsection{Trajectory optimization}
% \label{ssec:traj_opt}


% \subsection{Parameter estimation}
% \label{ssec:param_est}


\subsection{Interconnection and Damping Assignment Passivity-Based Control}
\label{ssec:ida-pbc}

Let $x \in \mathcal{X} \subset \mathbb{R}^{2n}$ denote the state of the robot. The state $x$ is
represented in terms of the generalized positions and momenta $x = (q, p)$. 
%
% It is known that for
% hyperregular Lagrangians, the Hamiltonian  and Lagrangian formulation of
% dynamics are equivalent~\citep{bullo2019geometric}. 
%
With $M$ denoting the symmetric, positive-definite mass
matrix, the Hamiltonian $H$ of the robot is expressed as 

\begin{equation}
    H(q,p) = \frac{1}{2} p^\top M^{-1}(q) p + V(q),
    \label{eq:system_hamiltonian}
\end{equation}

\noindent where $V(q)$ represents the potential energy. The system's equations
of motion can then be expressed as 

\begin{equation}
    \bmat{\dot{q} \\ \dot{p}} = \bmat{0 & I_n \\ -I_n & 0}\bmat{\nabla_qH \\
    \nabla_pH} + \bmat{0 \\ G(q)}u,
    \label{eq:hamiltonian_dynamics}
\end{equation}
%
where $G(q) \in \mathbb{R}^{n \times m}$ is the input matrix, $I_n$ denotes the
$n \times n$ identity matrix, and $u \in \mathbb{R}^m$ is the control input.
%
The system~\eqref{eq:hamiltonian_dynamics} is \textit{underactuated} if rank $G = m < n$.

The main idea of passivity-based control (PBC)~\citep{van2000l2} is to design
the control input $u \in \mathscr{U} \subseteq \mathbb{R}^m$ with the objective
of imposing a desired Hamiltonian $H_d: \mathbb{R}^{2n} \rightarrow \mathbb{R}$
on to the closed-loop system, rendering it passive (and consequently stable).
%
The choice of $H_d$ is such that some desired equilibrium $q^\star$ is stabilized.
%
In interconnection and damping assignment
(IDA-PBC)~\citep{ortega2002stabilization}, the closed-loop dynamics to chosen to
take on the port-controlled Hamiltonian (PCH) form:
%
\begin{equation}
  \bmat{\dot{q} \\ \dot{p}}
  =
  \bmat{J_d(q,p) - R_d(q,p)}
  \bmat{\nabla_q H_d \\ \nabla_p H_d}
  \label{eq:pch}
\end{equation}
%
where $J_d$ and $R_d$ are, respectively, the desired interconnection and damping
matrices:
%
\begin{align*}
    J_d &= -J_d^\top = \bmat{0 & M^{-1}M_d \\ -M_dM^{-1} & J_2(q,p)}, \\
    R_d &= R_d^\top = \bmat{0 & 0 \\ 0 & GK_vG^\top} \succeq 0, 
\end{align*}
%
and $J_2 = -J_2^\top$ serves as free parameters to ease solving the required
PDEs, which will later be elaborated, and $K_v \succ 0$ is the tunable gain for
damping injection.
%
The desired Hamiltonian $H_d$ is a quadratic function of the system momenta:
%
% the desired Hamiltonian is chosen to
% be a quadratic function of the system momenta as follows:
%
\begin{equation}
    H_d(q, p) = \frac{1}{2} p^\top M_d^{-1}(q) p + V_d(q),
    \label{eq:desired_hamiltonian}
\end{equation}
%
where $M_d(q) \succ 0$ is the closed-loop, positive definite mass matrix and
$V_d: \mathbb{R}^n \to \mathbb{R} $ is the closed-loop potential energy function that has an isolated
minimum at the desired equilibrium $q^\star$, i.e.
%
\begin{equation}
  q^\star = \underset{q}{\textrm{argmin}} \; \; V_d(q).
  \label{eq:argmin_Vd}
\end{equation}


The control law that achieves the objective of IDA-PBC is comprised of an energy-shaping term
and a damping injection term, i.e.
%
\begin{equation}
    u = u_{es}(q,p) + u_{di}(q,p)
    \label{eq:ida-pbc_control}.
\end{equation}
%
The energy-shaping term requires a solution to the equation
%
\begin{equation}
    Gu_{es} = \nabla_qH - M_dM^{-1} \nabla_qH_d + J_2M_d^{-1}p.
    \label{eq:Gues}
\end{equation}
%
If system is underactuated, $G$ is not invertible, and Equation~\eqref{eq:Gues}
cannot be uniquely solved. This leads to the constraints that must be
satisfied for any choice of $u_{es}$:
%
\begin{equation}
  G^\perp \left\{ \nabla_qH - M_dM^{-1} \nabla_qH_d + J_2M_d^{-1}p \right\} = 0,
  \label{eq:pde_main}
\end{equation}
%
where $G^\perp$ is the full-rank left annihilator of $G$, i.e. $G^\perp G = 0$.
Equation~\eqref{eq:pde_main} is a set of nonlinear partial differential
equations (PDE) parametrized by $M_d$, $V_d$, and $J_2$. The success of the
IDA-PBC approach hinges on the ability to solve this set of PDEs.


Solving the nonlinear PDEs is difficult in general. However, with $H_d$ of the
form in Equation~\eqref{eq:desired_hamiltonian}, the PDEs~\eqref{eq:pde_main}
can be separated into terms that depend on $p$ and terms that are independent of
$p$:
%
\begin{align}
  \label{eq:pde_1}
  G^\perp \left\{ \nabla_q\left(p^\top M^{-1} p\right) - M_dM^{-1} \nabla_q\left(p^\top M_d^{-1} p\right) + 2J_2M_d^{-1}p \right\} &= 0, \\
  \label{eq:pde_2}
  G^\perp \left\{ \nabla_qV - M_dM^{-1} \nabla_qV_d \right\} &= 0.
\end{align}
%
Equation~\eqref{eq:pde_1} has to be solved to obtain $M_d, J_2$. Once this is
obtained, Equation~\eqref{eq:pde_2} becomes a linear PDE that is simpler to
solve. The main difficulty lies within the solution of~\eqref{eq:pde_1}.
%
Once a solution is obtained, the energy shaping term of the control is
%
\begin{equation}
  u_{es} = \left(G^\top G\right)^{-1} G^\top \left(\nabla_qH - M_dM^{-1} \nabla_qH_d + J_2M_d^{-1}p\right).
  \label{eq:ues}
\end{equation}
%
With $K_v \succ 0$ a user-selected gain matrix, the damping injection term is given as
%
\begin{equation}
    u_{di} = -K_v G^\top \nabla_p H_d.
    \label{eq:udi}
\end{equation}
%


